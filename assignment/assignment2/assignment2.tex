\documentclass[12pt, a4paper, oneside]{ctexart}
\usepackage{amsmath, amsthm, amssymb, graphicx}
\title{assignment2}
\author{Xiaoma}
\date{2022.09.16}
\begin{document}
\maketitle
\subsection*{1.对于一个按升序排列的包含n个元素的有序数组A来说,HEAPSORT的时间复杂度
是多少?如果A是降序呢,请简要分析并给出结果。}
    降序排列:建堆时调用n次MAX-HEAPIFY(A,i),初始序列为降序,故
    每次时间复杂度为O(1),故建堆时间复杂度为O(n),排序时,调用(n-1)次MAX-HEAPIFY(A, i),
    若发生交换,MAX-HEAPIFY(A, i)还要进行logn次,故排序的时间复杂度为O(logn),整体时间
    复杂度为O(logn)。\\
    \par
    升序排列:建堆时调用n次MAX-HEAPIFY(A,i),每次都发生交换,时间复杂度为O(logn),排序时,调用
    (n-1)次MAX-HEAPIFY(A, i),若发生交换,MAX-HEAPIFY(A, i)还要进行logn次,
    故排序的时间复杂度为O(logn),整体时间复杂度为O(logn)。

\subsection*{2.快速排序:
    (a)假设快速排序每一层所做的划分比例都是$1-\alpha:\alpha$,其中$0<\alpha \leqslant 1/2$且是一个常数。试证明:在相应的递归树中,叶节点的最小深度大约是$-\lg n/ \lg \alpha$,最大深度约是 $-\lg n / (1-\alpha)$(无序考虑舍入问题)。
    (b)试证明:在一个随机输入数组上,对于任何常数$0<\alpha \leqslant 1/2$,PARTITION产生比$1-\alpha:\alpha$更平衡的划分的概率约为$1-2\alpha$。}
    \begin{itemize}
        \item [(a)] 若要求最小深度则选择划分比例小的一侧,不断迭代k次直至剩余
        一个元素,即$n\alpha ^{k}=1$,则$k=\frac{-\lg n}{\lg \alpha} $\\
        若要求最大深度则选择换分比例大的一侧,不断迭代k次直至剩余一个元素
        即$n(1-\alpha)^{k}=1$,则$k=\frac{-\lg n}{\lg(1-\alpha)}$\\
        \item [(b)] 若要产生比$1-\alpha:\alpha$更平衡的划分,那么划分点应该在$n\alpha$与$n(1-\alpha)$之间,又因为
        划分点的选取服从均匀分布,则$$\mathbf{P} =\frac{(1-\alpha)n-\alpha n}{n}=1-2\alpha $$
    \end{itemize}
\end{document}