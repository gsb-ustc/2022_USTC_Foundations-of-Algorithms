\documentclass[12pt, a4paper, oneside]{ctexart}
\usepackage{amsmath, amsthm, amssymb, bm, color, framed, graphicx, hyperref, mathrsfs}
\usepackage{enumerate}
\title{\textbf{assignment3}}
\author{Xiaoma}
\date{\today}
\linespread{1.5}
\definecolor{shadecolor}{RGB}{230, 245, 255}
\newcounter{problemname}
\newenvironment{problem}{\begin{shaded}\stepcounter{problemname}\par\noindent\textbf{题目\arabic{problemname}. }}{\end{shaded}\par}
\newenvironment{solution}{\par\noindent\textbf{解答. }}{\par}
\newenvironment{note}{\par\noindent\textbf{题目\arabic{problemname}的注记. }}{\par}

\begin{document}

\maketitle

\begin{problem}
下面的排序算法中哪些是稳定的:插入排序、归并排序、堆排序、快速排序和计数排序?给
出一个能使任何排序算法都稳定的方法。你所给出的方法带来的额外时间和空间开销是多少?
\end{problem}
\begin{solution}
    稳定排序:插入排序,归并排序,计数排序\\
    排序前为长度为n的序列设置长度为n的标签,若在排序时遇到值相同的数,则比较其标签,
    小标签在前。额外的时间复杂度为$O(n)$,空间复杂度为$O(n)$
\end{solution}

\begin{problem}
给定一个整数数组,其中不同的整数所包含的数字的位数可能不同。但该数组中,所有整数
中包含的总数字位数为 n。设计算法使其可以在 O(n) 时间内对该数组进行排序。
\end{problem}
\begin{solution}
    采用基数排序的方法,按位从低到高排序,RADIX-SORT使用计数排序,每位取值共有10
    种可能,设数组中数的位数为$d_{i}$,数组中共有x个数,则$\sum_{i = 1}^{x}d_{i}=n$,则数组每一位排序的时间复杂度为
    $O(x+10)$,则总时间复杂度为$O(d_{imax}(x+10))=O(n)$
\end{solution}

\begin{problem}
SELECT 算法 (找第 i 小的元素) 最坏情况下的比较次数 T(n) = Θ(n),但是其中的常数项
使非常大的。请对其进行优化,使其满足:\\
• 在最坏情况下的比较次数为 Θ(n)。\\
• 当 i 是小于 n/2 的常数时,最坏情况下只需要进行 n + O(log n) 次比较。

\end{problem}
\begin{solution}
    当$i\geq n/2$时,仍使用原算法。\\
    当$i\leq n/2$时,使用算法:
    \begin{enumerate}
        \item 设输入数组为$A[0 ,\dots ,n - 1]$,$m=\frac{n}{2}$,将数组分为两份,若$n$为奇数,则将第$n$个
              数暂时分出,此时得到两个等长数组$B_{l}[0 ,\dotsb ,m - 1], B_{r}[m ,\dotsb ,2m-1]$,对左右两侧的数组建立一一
              对应的映射$B_{l}[x]\leftrightarrow B_{r}[x+m] \quad x=0,\dotsb,m-1$,
              依次对每对映射对应的数字进行比较,若$B_{l}[x]>B_{r}[x+m]$,则交换两数。最终映射左侧的数小于映射
              右侧的数,保持该映射关系不变。
        \item 递归进行操作1直至$i>m/2$,注意在递归过程中,若映射对出现交换,上层分组仍应保持左侧小于
              右侧的规则,故可能进行多次交换。
        \item 对最终的$B_{r}$进行SELECT算法,得到$B_{r}$中最小的i个数,则$B_{l},B_{r}$中最小的i个数
              在两数组前i个数共2i个数中产生,对此2i个数进行SELECT算法,若n为奇数则将剩余的一个数放入,得到最小的i
              个数,返回上一层。
        \item 在上一层同理,对左右两数组前i个数进行SELECT得到该层最小的i个数,直至返回结束,得到A中最小的i
              个数,即得到目标值。
    \end{enumerate}
    时间复杂度分析:\\当$i<  n/2$时
    \begin{align*}
         & \quad COMPETITION(n)               \\
         & =n/2+COMPETITION(n/2)+T(2i)        \\
         & =n/2+n/2+f(T(2i)\log (n/2i))+T(2i) \\
         & =n + O(T(2i)\log(n/i))             \\
         & =n + O(\log n)
    \end{align*}
    当$i \geq n/2$时,
    \begin{align*}
         & \quad COMPETITION \\
         & = T(n)            \\
         & = \Theta (n)
    \end{align*}
\end{solution}
\end{document}