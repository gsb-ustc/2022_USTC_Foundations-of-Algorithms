\documentclass[12pt, a4paper, oneside]{ctexart}
\usepackage{amsmath, amsthm, amssymb, bm, color, framed, graphicx, hyperref, mathrsfs}
\usepackage{enumerate}
\usepackage{epstopdf}
\usepackage{float}
\title{\textbf{assignment3}}
\author{Xiaoma}
\date{\today}
\linespread{1.5}
\definecolor{shadecolor}{RGB}{230, 245, 255}
\newcounter{problemname}
\newenvironment{problem}{\begin{shaded}\stepcounter{problemname}\par\noindent\textbf{题目\arabic{problemname}. }}{\end{shaded}\par}
\newenvironment{solution}{\par\noindent\textbf{解答. }}{\par}
\newenvironment{note}{\par\noindent\textbf{题目\arabic{problemname}的注记. }}{\par}

\begin{document}

\maketitle

\begin{problem}
\end{problem}
\begin{solution}
    在没有切割成本时,钢条切割问题的状态转移方程为
    $$r_{n}=
    \begin{cases}
        p_{1} \quad n=1\\
        \max_{1\leq i \leq n}(p_{i} + r_{n-i}) \quad n > 1
    \end{cases}$$
    加入切割成本以后,状态转移方程变为
    $$r_{n}=
    \begin{cases}
        p_{1} \quad n=1\\
        \max(p_{n} + \max_{1 \leq i \leq n - 1}(p_{i} + r_{n - i} - c)) \quad n >1
    \end{cases}$$
    使用数组将不同长度的最大利润存储下来,自底向上计算钢条切割问题。
\end{solution}

\begin{problem}
    
\end{problem}
\begin{solution}
    
\end{solution}

\begin{problem}
    
\end{problem}
\begin{solution}
    \begin{enumerate}
        \item 对点集进行排序
        \item 从头开始遍历点集,设区间的左边界为$x_{i}$,
        然后去掉点集中满足$x_{i} \leq x_{j} \leq x_{i} + $
        的点$x_{j}$,直至遍历结束。
    \end{enumerate}
    \begin{itemize}
        \item 排序的时间复杂度为$O(n\log n)$
        \item 遍历点集的时间复杂度为$O(n)$
        \item 算法的总时间复杂度为$O(n\log n)$
    \end{itemize}
    证明:\\
    已知需要建立的区间为单位区间,设每个区间的左边界为$x_{1},x_{2},...,x_{m}$,
    则它们所在区间必然不相交,如果要包含点集中的所有点,则单位区间的数量至少要$m$个,
    所以该算法找到的是满足条件的最小区间。
\end{solution}

\begin{problem}
    
\end{problem}
\begin{solution}
    
\end{solution}

\begin{problem}
    
\end{problem}
\begin{solution}
    \begin{itemize}
        \item 每次尽可能拿最大面额的硬币,直到零钱总额为$n$。
    \end{itemize}
    证明:\\
    贪心算法求解问题的条件:
    \begin{enumerate}
        \item 贪心选择性质
        \item 最优子结构
    \end{enumerate}
    \begin{enumerate}
        \item 贪心选择性质:\\
        该问题可以通过局部最优解构造全局最优解,如果1美分的硬币达到5个,则可以换成1个5美分的硬币,其他情况同理,
        所以每次选择的零钱面额都会保证硬币数量最小,尽可能选择最大面额的硬币的局部最优解在
        全局最优解序列中。
        \item 最优子结构:\\
        每个问题的最优解都包含组成该问题的子问题的最优解,即大面额零钱兑换问题的最优解一定
        包含着子面额的零钱兑换问题的最优解。
    \end{enumerate}
    
\end{solution}
\end{document}