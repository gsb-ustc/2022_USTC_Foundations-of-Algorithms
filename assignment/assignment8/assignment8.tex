\documentclass[12pt, a4paper, oneside]{ctexart}
\usepackage{amsmath, amsthm, amssymb, bm, color, framed, graphicx, hyperref, mathrsfs}
\usepackage{enumerate}
\usepackage{epstopdf}
\usepackage{float}
\usepackage{framed}
\usepackage[ruled,vlined]{algorithm2e}
\title{\textbf{assignment8}}
\author{Xiaoma}
\date{\today}
\linespread{1.5}
\definecolor{shadecolor}{RGB}{230, 245, 255}
\newcounter{problemname}
\newenvironment{problem}{\begin{shaded}\stepcounter{problemname}\par\noindent\textbf{题目\arabic{problemname}. }}{\end{shaded}\par}
\newenvironment{solution}{\par\noindent\textbf{解答. }}{\par}
\newenvironment{note}{\par\noindent\textbf{题目\arabic{problemname}的注记. }}{\par}

\begin{document}

\maketitle

\begin{problem}
\end{problem}
\begin{solution}
\begin{enumerate}
    \item 若要计算$a^{n}$,已知当$n=0$时,$a^{n}=1$,当$n=1$时,$a^{n}=a$,当$n$为奇数时$a^{n} = (a^{\frac{n}{2}})^{2} \times a$,
    当$n$为偶数时,$a^{n}=(a^{\frac{n}{2}})^{2}$,故可将计算$a^{n}$拆分为以上子问题。子问题的最坏运行
    时间为$O(\log n)$。
    \item 合并过程仅需将子问题得到的数字相乘,最坏运行时间为$O(1)$。
    \item 由上述可知,原问题划分出的子问题是相同的,故只需要计算一个子问题然后进行合并,故不用重复其划分子问题和合并步骤。
    \item $a^{n}=(a^{\frac{n}{2}})(a^{\frac{n}{2}})a^{n\%2}=(a^{\frac{n}{4}}a^{\frac{n}{4}}a^{\frac{n}{2}\%2})(a^{\frac{n}{4}}a^{\frac{n}{4}}a^{\frac{n}{2}\%2})a^{n\%2}
    =\dotsb=(\Pi_{i=1}^{\frac{n}{2}}a)(\Pi_{i=1}^{\frac{n}{2}}a)a^{n\%2}$,故该算法一定能得到正确的计算解。
    \item 算法共需要递归$\log n$次,故时间复杂度为$O(\log n)$。
\end{enumerate}
\begin{algorithm*}
    \caption{myPow}
    \label{alg:algorithm}
    \KwIn{The base : a; The exponent : n;}
    \KwOut{$a^{n}$;}
    \BlankLine
    \If(){$n == 0$}{
        \Return{1};
    }
    \If(){$n == 1$}{
        \Return{a};
    }
    $temp = myPow(a, n/2);$
    \BlankLine
    \If(){$n \% 2 == 0$}{
        \Return{$temp * temp$};
    }
    \Return{temp * temp * a};
\end{algorithm*}
\end{solution}

\newpage
\begin{problem}
    
\end{problem}
\begin{solution}
    设两个数据库中数字的中位数分别为$A[a]$,$B[b]$,数据库A的首尾分别为$s_{1},d_{1}$,数据库B的首尾分别为$s_{2},d_{2}$,所以
    $a = (s_{1}+d_{1})/2, b = (s_{2}+d_{2})/2$。
    \begin{enumerate}
        \item 若$A[a]=B[b]$,则$A[a]$为两个数据库的中位数。
        \item 若$a<b$\begin{itemize}
            \item 若数据库A中元素个数为奇数,数据库A将$s_{1}$移动到$a$,数据库B将$d_{2}$移动到$b$
            \item 若数据库A中元素个数为偶数,数据库A将$s_{1}$移动到$a+1$,数据库B将$d_{2}$移动到$b$
        \end{itemize}
        \item 若$a>b$\begin{itemize}
            \item 若数据库A中元素个数为奇数,数据库A将$d_{1}$移动到$a$,数据库A将$s_{2}$移动到$b$
            \item 若数据库A中元素个数为偶数,数据库A将$d_{1}$移动到$a$,数据库A将$s_{2}$移动到$b+1$
        \end{itemize}
        \item 重复2,3直至1成立,或当两数据库的首尾范围均只包含一个数时,较小的数为所求中位数。
    \end{enumerate}
    \begin{algorithm*}
        \caption{mySearch}
        \label{alg:algorithm}
        \KwIn{Database A : A; Database B : B; The size of database : n;}
        \KwOut{The median of A and B;}
        \BlankLine
        s1 = 0, d1 = n - 1, s2 = 0, s2 = n - 1;
        \BlankLine
        \While(){$s1 != d1 || s2 != d2$}{
            a = (s1 + d1) / 2;\\
            b = (s2 + d2) / 2;\\
            \If(){$A[a] == B[b]$}{
                \Return{$A[a]$};
            }
            \If(){$A[a] < B[b]$}{
                \If(){$(s1 + d1) \% 2 == 0$}{
                    s1 = a;\\
                    d2 = b;
                }
                \Else(){
                    s1 = a + 1;\\
                    d1 = b;
                }
            }
            \Else(){
                \If(){$(s1 + d1) \% 2 == 0$}{
                    d1 = a;\\
                    s2 = b;
                }
                \Else(){
                    d1 = a;\\
                    s2 = b + 1;
                }
            }
        }
        \If(){A[a] > B[b]}{
            \Return{$B[b]$};
        }
        \Return{$A[a]$};
    \end{algorithm*}
    \textbf{每次迭代都舍弃一半的数据,故该算法的时间复杂度为$O(\log n)$。}
\end{solution}
\newpage
\begin{problem}
    
\end{problem}
\begin{solution}
    假设每次随机选择一个方向,若前进$i$($i$递增)米后,未找到宝藏,则回到出发点选择剩余未选择的方向,
    将每个方向都选择一次后,重新开始循环,直到找到宝藏。\\
    $$cost  (OPT) = 2^{j}+\varepsilon > 2^{j}$$
    $$cost (ON) = m(2+4+...+2^{j})+2^{j}+\varepsilon =(m+1)2^{j}+\varepsilon < (m+1)cost (OPT)$$
    则算法竞争比为$O(m)$。
\end{solution}
\end{document}